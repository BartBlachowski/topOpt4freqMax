\documentclass[preprint, 12pt, a4paper, onecolumn]{elsarticle}

%% The amssymb package provides various useful mathematical symbols
\usepackage{amsmath,tabu}
\usepackage{amssymb}
\usepackage{hyperref}
%\usepackage{refcheck}
\usepackage{statmath}
\usepackage{lineno}
%\usepackage{tikz}
%\usepackage{tikz-uml}
%\usepackage{pgfplots}
%\pgfplotsset{width=10cm,compat=1.9}
\usepackage{listings}
\usepackage[dvipsnames]{xcolor}
\usepackage{multirow}
\usepackage{makecell}
%\usetikzlibrary{positioning,calc}
\usepackage{subcaption}
%\usepackage{matlab-prettifier}
\usepackage{rotating}
\usepackage{array}
\usepackage{tikz}

\usepackage[export]{adjustbox}

\usetikzlibrary{fit}
\usetikzlibrary{shapes.geometric, arrows, positioning}

\usetikzlibrary{shapes.geometric, arrows, shadows}

\tikzstyle{startstop} = [rectangle, rounded corners, minimum width=3cm, minimum height=1cm,text centered, draw=black, fill=gray!30, drop shadow]
\tikzstyle{io} = [trapezium, trapezium left angle=70, trapezium right angle=110, minimum width=3cm, minimum height=1cm, text centered, draw=black, fill=blue!30]
\tikzstyle{process} = [rectangle, minimum width=3cm, minimum height=1cm, text centered, text width=3cm, draw=black, fill=blue!20]
\tikzstyle{decision} = [diamond, minimum width=3cm, minimum height=1cm, text centered, draw=black, fill=red!20]
\tikzstyle{arrow} = [thick,->,>=stealth]
\tikzstyle{line} = [draw, -latex']

%\documentclass[a4paper,11pt]{article}

%\usepackage[utf8]{inputenc}
%\usepackage[QX]{fontenc}   % potrzebne do polskich w utf8

\usepackage{graphicx}
\usepackage{array}
\usepackage{enumitem}
\usepackage{float}
\usepackage{algpseudocode}

\def\imwid{12cm}
\def\imwidh{6cm}
\def\imwidt{4cm}
\def\figd{figs}

\def\Nmax{N_{\textrm{max}}}
\def\Acirc{A_{\textrm{circ}}}
\def\Aenc{A_{\textrm{enc}}}
\def\Atot{A_{\textrm{tot}}}
\def\Nopen{N_{\textrm{open}}}
\def\Ncycl{N_{\textrm{cycl}}}
\newcommand{\ignore}[1]{}
\def\Npopul{N_{\textrm{popul}}}
\def\tmax{T_{\textrm{max}}}

\def\A{\emph{\textrm{Arm-Z}}} %Arm-Z in general

%\newcommand{\firstRevison}[1]{{\color{black} #1 \color{black}}}
\newcommand{\firstRevison}[1]{{\color{black} #1 \color{black}}}
\newcommand{\secondRevised}[1]{\textcolor{black}{#1}}

\makeatletter
\usepackage{tikz}
% #1 is a multiplier of fontsize for the minimum diameter of the circle
% #2 is the symbol to be circled.
\newcommand*\circled[2][1.6]{\tikz[baseline=(char.base)]{
    \node[shape=circle, draw, inner sep=1pt, 
        minimum height={\f@size*#1},] (char) {\vphantom{WAH1g}#2};}}
\makeatother

\journal{Computers \& Structures}

\usepackage[most]{tcolorbox}


% Define a custom style for your algorithms
\newtcolorbox{myalgo}{
    colback=gray!10, % 10% black (light grey)
    colframe=gray!50, % border color
    arc=0mm,          % square corners
    boxrule=0.5pt,    % border thickness
    left=5pt, right=5pt, top=5pt, bottom=5pt
}

\linenumbers

\begin{document}
\begin{frontmatter}

\title{Frequency maximization of beam structures using quasi-static approximation in topology optimization}

\author[IPPT]{Piotr Tauzowski}
\author[IPPT]{Andrzej Szczepańczyk}
\author[IPPT]{Bartłomiej Błachowski}

\address[IPPT]{Institute of Fundamental Technological Research, Polish Academy of Sciences \\
A. Pawinskiego 5B, 02-106 Warsaw, Poland}


\begin{abstract}
Topology optimization for maximizing natural frequencies is a critical tool for preventing structural resonance in engineering applications; however, its widespread adoption is often hindered by the prohibitive computational cost of repeated modal analyses. This paper presents an efficient method to determine the approximate topology that maximizes a target vibration frequency by replacing the standard generalized eigenvalue problem with the solution of a linear system of equations. While this approach shares a common objective with the recent two-step static optimization framework proposed by Yuksel and Yilmaz (2025), the current method distinguishes itself by identifying an approximate solution through a single static design-dependent problem. This refinement eliminates the need for the sequential iterative estimates required in previous models, further enhancing algorithmic efficiency.

The proposed method is validated on a simply supported beam benchmark against the approaches of Du and Olhoff (2007) and Yuksel and Yilmaz (2025). A second example on a clamped beam demonstrates multi-mode frequency control via a load-weighting parameter $\alpha$. Numerical results demonstrate that the proposed method achieves significant reductions in both run time and memory allocation. For a mesh size of $320 \times 40$, the method is approximately 8.6 times faster than the Olhoff formulation and 2.5 times faster than the Yuksel dynamic code, while requiring 21~MB of RAM compared to 163~MB (Olhoff) and 44~MB (Yuksel dynamic code). These findings indicate that the proposed single-step static approach is a robust and practical tool for the rapid design of vibration-resistant structures under limited computational resources.
\end{abstract}

\begin{keyword}
Topology optimization \sep Frequency maximization \sep Quasi-static approximation \sep Eigenvalue problem \sep Modal analysis
\end{keyword}

\end{frontmatter}

\section{Introduction}
In many engineering fields, including aerospace and high-precision machinery, the optimal design of structures with respect to natural frequencies is critical for ensuring operational safety and longevity. Structural resonance occurs when the external excitation frequency matches a structural natural frequency, potentially leading to large deformations, fatigue failures, or catastrophic fractures. Consequently, maximizing the fundamental natural frequency—the lowest frequency mode—is a primary objective in dynamic topology optimization, as it broadens the safe operating range and effectively prevents resonance in lightweight structures.

The foundational work in this field was significantly advanced by Du and Olhoff (2007)~\cite{Du2007} and further expanded by Olhoff and Du (2014)~\cite{Olhoff2014}, who introduced a robust framework utilizing a nested iterative solution procedure (often referred to as a double-loop method) to maximize simple and multiple eigenfrequencies. Their approach employs a "bound formulation" to handle the inherent non-smoothness and lack of differentiability that occurs when eigenvalues coalesce or switch orders during the optimization process. In this scheme, the generalized eigenvalue problem is solved in the main loop, while an inner loop solves a sub-optimization problem for design variable increments, ensuring that the fundamental frequency is moved as far as possible from external excitation intervals.

Beyond the traditional Solid Isotropic Material with Penalization (SIMP) method, several alternative frameworks have been developed to address frequency maximization. Huang et al.(2010)~\cite{Huang2010} developed the Bi-directional Evolutionary Structural Optimization (BESO) method, which utilizes discrete design variables and a modified interpolation scheme to effectively eliminate artificial localized modes in low-density regions. Another significant approach is the Level Set Method (LSM), as explored by Xia et al.(2011)~\cite{Xia2011} and Shu et al. (2014)~\cite{Shu2014}, which provides an explicit representation of structural boundaries and utilizes derivatives to handle multiple eigenvalues while maintaining the topological nature of the design. More recently, Huang et al.(2025)~\cite{Huang2025} provided a framework for Moving Morphable Component (MMC) based optimization, where structural layouts are optimized by updating the geometric parameters of morphable building blocks, offering crisp boundaries and a reduced number of design variables.

To improve computational efficiency, Yuksel and Yilmaz (2025)~\cite{Yuksel2025} proposed a fast algorithm that avoids the high cost of eigenvalue calculations by transforming the dynamic problem into two sequential static optimization steps. The first step solves a compliance minimization problem under a static point load to obtain an initial estimate of the fundamental mode shape, while the second step refines this through an iterative process utilizing design-dependent inertial loads proportional to the structure's mass distribution. By leveraging Rayleigh’s principle, this method allows the fundamental frequency to be maximized using efficient static solvers, resulting in significant savings in memory usage and run times compared to standard dynamic approaches.

Finally, researchers have addressed a diverse array of other complex dynamic problems. These include the optimization of structures under stationary stochastic dynamic loads using Lyapunov equations proposed by Gomez and Spencer (2019)~\cite{Gomez2019}, and the imposition of frequency band constraints through modified Heaviside functions by Li et al.(2021)~\cite{Li2021}. Further studies by Wu et al.(2024)~\cite{Wu2024} and Pozzi et al.(2025)~\cite{Pozzi2025} have focused on structures exhibiting nonlinear structural dynamics or frequency-dependent material properties. Additionally, advancements have been made in multiscale composite structures for frequency maximization by Lee et al.(2024)~\cite{Lee2024}, the optimization of extruded beams using XFEM by Marzok and Waisman (2024)~\cite{Marzok2024}, and the handling of hybrid uncertainties in dynamic problems by Wu (2022)~\cite{Wu2022}. Numerical improvements such as benchmarking solvers for topology optimization by Rojas-Labanda and Stolpe (2015)~\cite{RojasLabanda2015} and quadratic programming (QP) modifications for self-weight problems by Munro (2024)~\cite{Munro2024} continue to refine the efficiency of these dynamic frameworks.

In this paper, we present an efficient method for determining an approximate topology that maximizes a target natural frequency. In contrast to conventional approaches that require solving the generalized eigenvalue problem at every optimization step, such as Du and Olhoff (2007)~\cite{Du2007}, the proposed method replaces repeated modal analysis with the solution of a linear system of equations. This approach is similar in principle to the work of Yuksel and Yilmaz (2025)~\cite{Yuksel2025}; however, our method distinguishes itself by determining an approximate solution through a single static problem rather than a two-step procedure, thereby further enhancing computational efficiency.

The remainder of this paper is organized as follows. In Section 2, the theoretical framework for topology optimization for frequency maximization is established; specifically, Section 2.1 provides a review of existing approaches for fundamental frequency optimization, while Section 2.2 details the step-by-step procedure based on the approach proposed in this work. Section 3 presents two numerical examples: a simply supported beam used to validate the method against the Olhoff and Yuksel approaches, and a clamped beam used to demonstrate multi-mode frequency control via a load-weighting parameter. In Section 4, a comprehensive discussion on various aspects of computational implementation is provided; Section 4.1 highlights the algorithmic differences between the implemented and original Olhoff formulations, while Section 4.2 evaluates the algorithm, efficiency, and practical merits of the proposed approach. Finally, Section 5 summarizes the findings and draws concluding remarks.

\section{Topology optimization for frequency maximization}

This section describes the proposed efficient method for finding the optimal topology based on the fundamental vibration frequency. To better relate existing methods, a subsection describing the computational complexity of these methods is presented. It is also worth noting that the proposed approach is independent of the specific optimization algorithm. However, for the sake of numerical comparison, the effectiveness of the proposed method will be presented in the context of a popular strategy known in the literature as SIMP.

\subsection{Existing approaches for fundamental frequency optimization}
Topology optimization for frequency maximization can be formulated as a max--min problem as follows:
\begin{equation} \label{eqn:minmax}
\max_{x_1,\ldots,x_{n_e}} \; \Bigg\{ \min_{j=1,\ldots,J} \left\{ \omega_j^2 \right\} \Bigg\}
\end{equation}
subject to
\begin{align*}
\mathbf{K}\boldsymbol{\phi}_j = \omega_j^2 \mathbf{M}\boldsymbol{\phi}_j &, \qquad j = 1,\ldots,J, \\[4pt]
\boldsymbol{\phi}_j^{\mathrm T} \mathbf{M} \boldsymbol{\phi}_k = \delta_{jk} &, \qquad k,j = 1,\ldots,J, \\[4pt]
\sum_{e=1}^{n_e} x_e V_e - \alpha V_0 \le 0 &, \\[4pt]
0 < x_{\min} \le x_e \le 1 &, \qquad e = 1,\ldots,n_e.
\end{align*}
 
In these equations, $\omega_j$ is the $j$-th eigenfrequency and $\boldsymbol{\phi}_j$ the corresponding eigenvector, and 
$\mathbf{K}$ and $\mathbf{M}$ are the symmetric and positive definite stiffness and mass matrices of the finite-element-based generalized structural eigenvalue problem in the constraint. The $J$ candidate eigenfrequencies considered will all be real and can be numbered such that $0 < \omega_1 \le \omega_2 \le \cdots \le \omega_J,$ and it will be assumed that the corresponding eigenvectors are $\mathbf{M}$~-~orthonormalized, where $\delta_{jk}$ is 
Kronecker’s delta.
 
In the problem~(\ref{eqn:minmax}), the symbol $n_e$ denotes the total number of finite elements in the admissible design domain. The design variables $x_e$, $e=1,\ldots,n_e$, represent the volumetric material densities of the finite elements, and last constraint specifies lower and upper limits $x_{\min}$ and $1$ for $x_e$. To avoid singularity of the stiffness matrix, $x_{\min}$ is not taken as zero but as a small positive value, e.g.\ $x_{\min} = 10^{-3}$.
Symbol $\alpha$ defines the volume fraction, where $V_0$ is the volume of the admissible design domain.

Before proposing a new method for topological optimization with respect to the vibration frequency, two existing approaches will be briefly discussed.

The first approach has been proposed by Du and Olhoff (2007)~\cite{Du2007}. This approach provides a robust framework for the topology optimization of freely vibrating continuum structures using the SIMP method. It specifically addresses the challenge of maximizing simple and multiple eigenfrequencies, as well as frequency gaps, to avoid structural resonance. A central feature of the approach is the use of a "bound formulation," which introduces a scalar variable that acts as both the objective function and a variable lower bound for the targeted eigenfrequencies. This formulation is designed to handle the non-smoothness and lack of differentiability that occur when eigenfrequencies coalesce or switch orders during the design process, ensuring that the optimization problem remains mathematically tractable.
The numerical solution is implemented through a nested (double-loop) iterative procedure. In the main (outer) loop, the generalized eigenvalue problem is solved using finite element analysis to establish the current eigenfrequencies and eigenvectors and to detect possible multiplicity. The inner loop solves a sub-optimization problem to determine the optimal increments of design variables — using the increments as unknowns to simplify the treatment of multiple eigenvalues — typically solved via mathematical programming such as the Method of Moving Asymptotes (MMA). To ensure stable convergence and avoid common numerical artifacts, the method incorporates mesh-independent sensitivity filtering and utilizes a modified interpolation scheme with a high penalization factor for mass in low-density regions to eliminate artificial localized modes.

The second approach has been introduced by Yuksel and Yilmaz (2025)~\cite{Yuksel2025}. They proposed an efficient algorithm designed to maximize the fundamental natural frequency of structural beams while avoiding the high computational cost of eigenvalue and eigenvector calculations at every iteration. Their method leverages Rayleigh’s principle to transform the dynamic topology optimization problem into two sequential static topology optimization steps. This approach is built upon the premise that a structure's fundamental frequency is proportional to its stiffness when considering its mass distribution; thus, by maximizing stiffness for inertial forces associated with free vibration in the fundamental mode, the frequency can be maximized indirectly.
The implementation begins with a first static optimization step where a point load is applied to identify a suitable initial estimate of the fundamental mode shape that adheres to the prescribed boundary conditions. In the second step, this estimate is refined through an iterative compliance minimization process utilizing design-dependent inertial loads ($f_d = M \hat{u}_d$) that are proportional to the structural mass distribution and updated as the topology evolves. This strategy results in significant computational savings, with reported run times for large meshes being less than 1\% of those required by standard dynamic topology optimization methods such as described above approach by Du and Olhoff. Furthermore, Yuksel and Yilmaz introduced a hybrid method that utilizes this fast algorithm to generate an initial design guess, which is then finalized using standard dynamic procedures to balance efficiency with maximum performance.

\subsection{Step-by-step procedure based on proposed approach}
In this subsection, we present an efficient method that allows us to determine the approximate topology that maximizes a chosen vibration frequency. Compared to the approaches described in the previous subsection, the proposed method avoids the repeated solving of the eigenvalue problem at each optimization step (Du and Olhoff~(2007)~\cite{Du2007}) by replacing the solution of the eigenvalue problem with a solution of a linear system of equations, similarly to the work of Yuksel and Yilmaz~(2025)~\cite{Yuksel2025}. However, unlike the latter work, it determines an approximate solution in a single static problem instead of a two-step solution, which further increases the efficiency of the proposed method.

\textbf{Step 1.} Initialize the procedure solving eigenvalue problem for the whole design domain $x_{e}=1, e=1,2,...,n_{e}$
\begin{equation}
(\mathbf{K}-\omega_{i}^{2}\mathbf{M})\boldsymbol{\phi}_{i}=0, \quad i=1
\end{equation}

\begin{figure}[H]
    \center{ 
        \includegraphics[width=\textwidth]{figures/methods/algorithm_design_domain_mode.png}
    }
    \caption{First mode shape determined for the whole design domain.}
    \label{fig:first}
\end{figure}

\textbf{Step 2.} Extract first mode shape to formulate load vector
\begin{equation}
\mathbf{f}(\mathbf{x})=\omega_{i}^{2}\mathbf{M}(\mathbf{x})\boldsymbol{\phi}_{i}
\end{equation}
in the above formula eigenpair $(\omega_i, \boldsymbol{\phi}_i)$ remains constant during the whole optimization process. The load vector will be updated only through modification in mass matrix $\mathbf{M}(\mathbf{x}).$

\begin{figure}[H]
    \center{ 
        \includegraphics[width=\textwidth]{figures/methods/algorithm_modal_forces_domain.png}
    }
    \caption{Inertial forces corresponding to the 1st mode shape of the design domain.}
    \label{fig:second}
\end{figure}

\textbf{Step 3.} Apply load vector obtained in Step 2 to static design-dependent optimization problem under volume constraint

\begin{equation} \label{eqn:SIMP}
\max_{x_1,\ldots,x_{n_e}} c(\mathbf{x}) = \mathbf{q}^{\textrm{T}}\mathbf{K}(\mathbf{x})\mathbf{q}
\end{equation}
subject to
\begin{align*}
\mathbf{K}(\mathbf{x})\mathbf{q}=\mathbf{f}(\mathbf{x}) &,\\[4pt]
\sum_{e=1}^{n_e} x_e V_e = \alpha V_0 &, \\[4pt]
0 < x_{\min} \le x_e \le 1 &, \qquad e = 1,\ldots,n_e.
\end{align*}

\begin{figure}[H]
    \center{ 
        \includegraphics[width=\textwidth]{figures/methods/algorithm_modal_forces_topology.png}
    }
    \caption{Inertial forces for the 1st mode shape of the final topology.}
    \label{fig:third}
\end{figure}

\textbf{Step 4.} Perform optimization from Step 3 modifying at each iteration the load vector making it depend on the current values of design variables $\mathbf{M}(\mathbf{x})$. Continue this static optimization until convergence is achieved.

\begin{figure}[H]
    \center{ 
        \includegraphics[width=\textwidth]{figures/methods/algorithm_optimal_topology.png}
    }
    \caption{Suboptimal mass distribution maximizing fundamental frequency.}
    \label{fig:forth}
\end{figure}

\section{Numerical examples}

Two benchmark structures are examined. The first is a simply supported beam, used to compare the proposed method against the Olhoff and Yuksel approaches. The second is a clamped beam, used to illustrate multi-mode frequency control via the load-weighting parameter $\alpha$.

\subsection{Simply supported beam}
The benchmark problem follows the simply supported beam configuration studied by Du and Olhoff~\cite{Du2007} and reproduced in Yuksel and Yilmaz~\cite{Yuksel2025}.  The admissible design domain is a rectangular beam of length $L = 8$~m and height $H = 1$~m, discretized into $400 \times 50$ four-node quadrilateral plane-stress finite elements, yielding $n_\mathrm{e} =
20{,}000$ design variables and $n_\mathrm{dof} = 40{,}902$ degrees of freedom.
The material is isotropic with Young's modulus $E_0 = 10^7$~Pa, Poisson's ratio $\nu = 0.3$, and mass density $\rho_0 = 1~\mathrm{kg/m}^3$.  A void pseudo-material with $E_\mathrm{min} = 10^{-9}E_0$ and $\rho_\mathrm{min} =
10^{-9}\rho_0$ is used to prevent singularity of the global stiffness and mass matrices.  The SIMP penalization exponent is $p = 3$ for both stiffness and mass.  Boundary conditions follow Fig.~2(a) of Olhoff and Du~\cite{Du2007}: both end edges are simply supported by pin constraints applied at the mid-height node of each edge (neutral axis), leaving all other edge nodes free.  No concentrated nonstructural mass is attached.  The prescribed volume fraction is $\alpha = 0.5$, and a sensitivity filter with
radius $r_\mathrm{min} = 2$ elements is applied in all three implementations to suppress checkerboard patterns and ensure mesh-independent topologies.
Convergence of the Yuksel dynamic code and the proposed approach is declared when the maximum element-wise density change satisfies $\max|\Delta x_e| < 10^{-3}$; the Olhoff implementation uses an additional grayness criterion ($\bar{g} < 0.05$) in conjunction with the projection continuation schedule.
\begin{figure}[H]
    \center{ 
        \includegraphics[width=\textwidth]{figures/results/Olhoff_400x50.png}
        \caption{Optimal topology obtained using Olhoff approach ($\omega_1 = 174.2 \, \textrm{rad/s}$).}
        \includegraphics[width=\textwidth]{figures/results/Yuksel_400x50.png}
        \caption{Optimal topology obtained using Yuksal approach ($\omega_1 = 160.5 \, \textrm{rad/s}$).}
        \includegraphics[width=\textwidth]{figures/results/OurApproach_400x50.png}
        \caption{Optimal topology obtained using proposed approach ($\omega_1 = 159.6 \, \textrm{rad/s}$). The same mesh resolution 400 x 50 was used in all three approaches.}
    }    
    \label{fig:firstEx}
\end{figure}
The optimized topologies obtained by the three approaches at the $400{\times}50$ resolution are shown in Figs.~5--7.  All three designs share the same global load-path architecture: two stiff flanges connected by a system of diagonal bracing members, consistent with classical beam-frequency optimization results reported in the literature~\cite{Du2007,Yuksel2025}.  The Olhoff
approach achieves the highest fundamental frequency $\omega_1 = 174.2$~rad/s and produces a topology in which the first two eigenfrequencies are bimodal at the optimum---a known characteristic of the bound formulation for simply supported beams~\cite{Du2007}.  The Yuksel dynamic code and the proposed approach yield $\omega_1 = 160.5$~rad/s and $\omega_1 = 159.6$~rad/s respectively, a mutual difference of less than $0.6\,\%$, while the gap with respect to the Olhoff result is approximately $8.4\,\%$.  This discrepancy reflects the intrinsic approximation shared by both static-load-based methods:
fixing $\omega_1$ and $\boldsymbol{\varphi}_1$ from an intermediate or initial design introduces a bias that prevents these methods from reaching the true eigenvalue maximum, whereas the Olhoff bound formulation directly maximizes the smallest eigenfrequency at every iteration.  

In terms of computational resources, the Olhoff implementation requires $172.6$~s and $384$~MB of RAM to execute $296$ outer iterations, whereas the Yuksel dynamic code converges in $700$ iterations over $48.3$~s with $123$~MB, and the proposed approach terminates in $500$ iterations in $40.0$~s using $112$~MB.  The proposed method thus achieves a runtime $4.3\times$ shorter than the Olhoff implementation and $1.2\times$ shorter than the Yuksel dynamic code while producing a topology and fundamental frequency that are practically indistinguishable from the Yuksel result, confirming the practical competitiveness of the proposed approach for engineering design scenarios in which computational turnaround time and implementation simplicity are primary concerns.
\begin{table}[htbp]
\centering
\caption{Performance comparison of optimization approaches for different mesh sizes.}
\renewcommand\arraystretch{1.2}
\begin{tabular}{lcccc}
\Xhline{1.2pt}
Method & Iterations & Run time (s) & Run time/iter (s) & Max RAM (MB) \\
\Xhline{1.2pt}
 
\multicolumn{5}{c}{\textbf{Mesh: $160 \times 20$}} \\
\Xhline{0.8pt}
Olhoff   & 337 & 27.1  & 0.08 & 38  \\
Yuksel   & 622 & 7.3   & 0.01 & 3   \\
Proposed & 500 & 6.1   & 0.01 & 4   \\
\Xhline{1.2pt}
 
\multicolumn{5}{c}{\textbf{Mesh: $240 \times 30$}} \\
\Xhline{0.8pt}
Olhoff   & 313 & 64.2  & 0.21 & 79  \\
Yuksel   & 662 & 18.1  & 0.03 & 2   \\
Proposed & 500 & 13.8  & 0.03 & 4   \\
\Xhline{1.2pt}
 
\multicolumn{5}{c}{\textbf{Mesh: $320 \times 40$}} \\
\Xhline{0.8pt}
Olhoff   & 304 & 114.4 & 0.38 & 163 \\
Yuksel   & 700 & 33.1  & 0.05 & 44  \\
Proposed & 259 & 13.2  & 0.05 & 21  \\
\Xhline{1.2pt}
 
\multicolumn{5}{c}{\textbf{Mesh: $400 \times 50$}} \\
\Xhline{0.8pt}
Olhoff   & 296 & 172.6 & 0.58 & 384 \\
Yuksel   & 700 & 48.3  & 0.07 & 123 \\
Proposed & 500 & 40.0  & 0.08 & 112 \\
\Xhline{1.2pt}
 
\end{tabular}
\end{table}


\subsection{Clamped beam}\label{subsec:clampedBeam}
The second numerical example considers a beam clamped at both ends. The admissible design domain is a rectangle of length $L=8$~m and height $H=1$~m (unit thickness), discretized with a $400\times 50$ mesh of four-node plane-stress elements ($n_e=20{,}000$). Both end edges are fully clamped, i.e.\ $u_x=u_y=0$ at $x=0$ and $x=L$. The material model is isotropic with $E_0=10^7$~Pa, $\nu=0.3$, and $\rho_0=1.0$~kg/m$^3$; a void pseudo-material with $E_\mathrm{min}=10^{-6}E_0$ and $\rho_\mathrm{min}=10^{-6}$~kg/m$^3$ is used. SIMP with penalization $p=3$ is applied together with a sensitivity filter of radius $r_{\min}=2$ elements ($0.04$~m). The MMA optimizer is used with move limit $0.2$ and convergence tolerance $10^{-3}$. The prescribed volume fraction is $V^\ast/V_0=0.5$.

Let $(\omega_1,\boldsymbol{\Phi}_1)$ and $(\omega_2,\boldsymbol{\Phi}_2)$ denote the first two eigenpairs of the clamped design domain. We focus on five topologies obtained from two harmonic inertial loads
\begin{equation}
\mathbf{f}_{\alpha}(\mathbf{x})=\left\{\alpha\,\omega_{1}^{2}\mathbf{M}(\mathbf{x})\boldsymbol{\Phi}_{1},\;\left(1-\alpha\right)\omega_{2}^{2}\mathbf{M}(\mathbf{x})\boldsymbol{\Phi}_{2}\right\},
\end{equation} where $\alpha \in \{0,\;0.25,\;0.5,\;0.75,\;1.0\}$.
The corresponding layouts are shown in Fig.~\ref{fig:secondEx} (top to bottom: $\alpha=1.0,\;0.75,\;0.5,\;0.25,\;0$), and the first two eigenfrequencies are summarized in Table~\ref{tab:clampedBeamFreq}.
\begin{table}[H]
\centering
\caption{First two eigenfrequencies for the five topologies in the clamped-beam example.}
\label{tab:clampedBeamFreq}
\begin{tabular}{ccc}
\Xhline{1.2pt}
$\alpha$ & $\omega_1$ [rad/s] & $\omega_2$ [rad/s] \\
\Xhline{1.2pt}
1.00 & 275.28 & 365.78 \\
0.75 & 347.70 & 360.36 \\
0.50 & 318.85 & 553.35 \\
0.25 & 254.04 & 422.56 \\
0.00 & 113.30 & 340.43 \\
\Xhline{1.2pt}
\end{tabular}
\end{table}
The results reveal a non-monotonic trade-off between the two modes (Table~\ref{tab:clampedBeamFreq}). The highest $\omega_1$ is achieved not at $\alpha=1.0$ (pure mode-1 excitation) but at $\alpha=0.75$, where $\omega_1$ and $\omega_2$ are nearly equal — the mixed load steers the optimizer toward a stiffer, balanced topology. This is consistent with the known limitation of fixing the mode shape from the initial solid domain: for clamped boundary conditions the solid-domain mode shape is a poorer proxy for the optimal topology's mode shape than for simply supported conditions, and blending in a second-mode component partially corrects the load direction, yielding a higher $\omega_1$ than the nominally pure mode-1 case. As $\alpha$ decreases further, $\omega_1$ drops sharply (by about $67\,\%$ from its peak), while $\omega_2$ first rises to its maximum at $\alpha=0.5$ before falling again. The pure mode-2 topology ($\alpha=0.0$) therefore yields neither the highest $\omega_2$ nor the lowest penalty on $\omega_1$. The intermediate layouts ($\alpha=0.5$ and $\alpha=0.75$) provide the best compromise between the two frequencies. The topology evolution in Fig.~\ref{fig:secondEx} confirms this modal competition: material is progressively redistributed from a single-arch mode-1 bracing pattern (top) toward multi-cell configurations that stiffen higher modes.

\begin{figure}[H]
\centering
\begin{subfigure}{\textwidth}
    \centering
    \includegraphics[width=\textwidth]{figures/results/alpha_1.png}
    \caption{$\alpha=1.00$: $\omega_1=275.28$~rad/s, $\omega_2=365.78$~rad/s}
\end{subfigure}
\begin{subfigure}{\textwidth}
    \centering
    \includegraphics[width=\textwidth]{figures/results/alpha_2.png}
    \caption{$\alpha=0.75$: $\omega_1=347.70$~rad/s, $\omega_2=360.36$~rad/s}
\end{subfigure}
\begin{subfigure}{\textwidth}
    \centering
    \includegraphics[width=\textwidth]{figures/results/alpha_3.png}
    \caption{$\alpha=0.50$: $\omega_1=318.85$~rad/s, $\omega_2=553.35$~rad/s}
\end{subfigure}
\begin{subfigure}{\textwidth}
    \centering
    \includegraphics[width=\textwidth]{figures/results/alpha_4.png}
    \caption{$\alpha=0.25$: $\omega_1=254.04$~rad/s, $\omega_2=422.56$~rad/s}
\end{subfigure}
\begin{subfigure}{\textwidth}
    \centering
    \includegraphics[width=\textwidth]{figures/results/alpha_5.png}
    \caption{$\alpha=0.00$: $\omega_1=113.30$~rad/s, $\omega_2=340.43$~rad/s}
\end{subfigure}
\caption{Optimal topologies for the clamped-beam example at five values of the load-weighting parameter $\alpha$. Mesh $400\times 50$, volume fraction $0.5$.}
\label{fig:secondEx}
\end{figure}



\subsection{Building example}\label{subsec:building}

The third numerical example is inspired by the lateral-frequency design of a building-like planar frame. The design domain is a rectangle of width $W=5$~m and height $H=15$~m (unit thickness), discretized with an $80\times 240$ mesh of four-node plane-stress elements ($n_e=19{,}200$). The base edge ($y=0$) is fully clamped ($u_x=u_y=0$); all other edges are traction-free. Passive (permanently solid) regions enforce a three-story structural frame: two vertical columns of width $0.1$~m run the full height along both side edges, and horizontal floor/ceiling plates of thickness $0.1$~m are placed at $y=5$~m, $y=10$~m, and $y=15$~m, representing the permanent structural frame elements (see Fig.~\ref{fig:building}(a)). The optimizable material fills the three story panels between the floor plates. Material parameters ($E_0=10^7$~Pa, $\nu=0.3$, $\rho_0=1$~kg/m$^3$), void material ($E_\mathrm{min}=10^{-6}E_0$, $\rho_\mathrm{min}=10^{-6}$~kg/m$^3$), SIMP penalization ($p=3$), and sensitivity filter radius ($r_{\min}=2$ elements) match the clamped-beam example. The volume fraction is lower at $V^\ast/V_0=0.3$. The MMA optimizer with move limit $0.2$ and convergence tolerance $2\times 10^{-3}$ is used.

Five topologies are generated using the same mixed inertial load as in Section~\ref{subsec:clampedBeam}, with $\alpha \in \{0,\;0.25,\;0.5,\;0.75,\;1.0\}$. The load shapes $\boldsymbol{\Phi}_1$ and $\boldsymbol{\Phi}_2$ are the first two eigenvectors of the fully solid design domain, whose corresponding eigenfrequencies are $\omega_1^{(0)}=19.84$~rad/s and $\omega_2^{(0)}=90.94$~rad/s respectively. The optimized layouts are shown in Fig.~\ref{fig:building} and the resulting eigenfrequencies are summarized in Table~\ref{tab:buildingFreq}.

\begin{table}[H]
\centering
\caption{First two eigenfrequencies for the five topologies in the building example.}
\label{tab:buildingFreq}
\begin{tabular}{ccc}
\Xhline{1.2pt}
$\alpha$ & $\omega_1$ [rad/s] & $\omega_2$ [rad/s] \\
\Xhline{1.2pt}
1.00 & 71.36 & 74.27 \\
0.75 & 73.91 & 76.93 \\
0.50 & 66.93 & 78.62 \\
0.25 & 52.04 & 63.54 \\
0.00 & 33.56 & 50.82 \\
\Xhline{1.2pt}
\end{tabular}
\end{table}

The building example exhibits a richer modal interaction than the beam cases owing to the passive frame, the $3:1$ height-to-width aspect ratio, and the lower volume fraction. Across all $\alpha$ values the optimizer raises $\omega_1$ well above its solid-domain baseline of $19.84$~rad/s, with the best topology ($\alpha=0.75$) achieving $\omega_1=73.91$~rad/s --- a $3.7\times$ improvement. The non-monotonic $\omega_1$ behaviour persists: the highest fundamental frequency ($\omega_1=73.91$~rad/s) is achieved at $\alpha=0.75$ rather than at $\alpha=1.0$ ($\omega_1=71.36$~rad/s), confirming the approximation-correction mechanism identified in the clamped-beam example. Notably, at $\alpha=0.75$ the mixed load Pareto-dominates the pure mode-1 load: $\omega_2=76.93$~rad/s exceeds $\omega_2=74.27$~rad/s at $\alpha=1.0$, a simultaneous improvement of both frequencies not observed in the clamped-beam case. The second eigenfrequency peaks at $\alpha=0.50$ ($\omega_2=78.62$~rad/s). Beyond $\alpha=0.50$, both frequencies decline sharply as $\alpha$ approaches zero, reaching $\omega_1=33.56$~rad/s at $\alpha=0.0$ --- a reduction of more than $54\,\%$ relative to the $\alpha=0.75$ peak. The topology evolution (Fig.~\ref{fig:building}) reflects these frequency shifts: X-braced inter-story infill at $\alpha=1.0$ transitions toward more complex multi-diagonal configurations as the second mode is progressively emphasized.

\begin{figure}[H]
\centering
% Row 1
\begin{subfigure}[t]{0.45\textwidth}
    \centering
    \includegraphics[height=7cm]{figures/results/building_model.png}
    \caption{Design domain: clamped at base, solid side columns and floor plates (passive elements).}
\end{subfigure}
\hfill
\begin{subfigure}[t]{0.45\textwidth}
    \centering
    \includegraphics[height=7cm]{figures/results/building_alpha_1.png}
    \caption{$\alpha=1.00$: $\omega_1=71.36$~rad/s, $\omega_2=74.27$~rad/s}
\end{subfigure}
% Row 2
\begin{subfigure}[t]{0.45\textwidth}
    \centering
    \includegraphics[height=7cm]{figures/results/building_alpha_2.png}
    \caption{$\alpha=0.75$: $\omega_1=73.91$~rad/s, $\omega_2=76.93$~rad/s}
\end{subfigure}
\hfill
\begin{subfigure}[t]{0.45\textwidth}
    \centering
    \includegraphics[height=7cm]{figures/results/building_alpha_3.png}
    \caption{$\alpha=0.50$: $\omega_1=66.93$~rad/s, $\omega_2=78.62$~rad/s}
\end{subfigure}
% Row 3
\begin{subfigure}[t]{0.45\textwidth}
    \centering
    \includegraphics[height=7cm]{figures/results/building_alpha_4.png}
    \caption{$\alpha=0.25$: $\omega_1=52.04$~rad/s, $\omega_2=63.54$~rad/s}
\end{subfigure}
\hfill
\begin{subfigure}[t]{0.45\textwidth}
    \centering
    \includegraphics[height=7cm]{figures/results/building_alpha_5.png}
    \caption{$\alpha=0.00$: $\omega_1=33.56$~rad/s, $\omega_2=50.82$~rad/s}
\end{subfigure}
\caption{Building example: design domain schematic and optimal topologies for five values of $\alpha$. The domain is clamped at the base; passive (solid) elements form the side columns and floor plates. Mesh $80\times 240$, volume fraction $0.3$.}
\label{fig:building}
\end{figure}


\section{Discussion on various aspects of computational implementation}

This section of the paper will discuss aspects related to the computer implementation of the discussed methods for maximizing vibration frequency. To provide a fair comparison of the computational results and to enable the Reader to reproduce the results presented in the paper, all examples were calculated using the SIMP-based implementation developed by the Authors. All the numerical codes discussed are publicly available as a GitHub repository (\href{https://github.com/BartBlachowski/topOpt4freqMax}{https://github.com/BartBlachowski/topOpt4freqMax}).

This repository contains: a double-loop topological optimization method based on the work of Du and Olhoff (2007), a two-stage static method described by Yuksel and Yilmaz (2025), and the method proposed in this paper. All approaches discussed in this study were tested in the MATLAB 2025b environment.

% Algorithmic analysis of the implemented Olhoff approach relative to the original
% formulation of Olhoff and Du (2014).  This fragment is intended for \input{} into
% a journal manuscript; it contains no \documentclass or \begin{document}.
%
% References assumed available in the parent document:
%   \cite{OlhoffDu2014}  -- Olhoff & Du, CISM 2014 (978-3-7091-1643-2_11)
%   \cite{YukselYilmaz2025} -- Yuksel & Yilmaz, Eng.\ Comput.\ 2025
%   \cite{Svanberg1987}  -- Svanberg, IJNME 1987

\subsection{Algorithmic differences between the implemented and original Olhoff formulations}

The original formulation of Olhoff and Du~\cite{Olhoff2014} casts the fundamental-eigenfrequency maximization problem as a max-min bound problem in which a scalar variable $\beta \leq \omega_j^2$, $j = n, n{+}1, \ldots, J$, plays simultaneously the role of objective and lower bound for the $J$ candidate eigenfrequencies.  The computational procedure (Fig.~1 of that reference) consists of a \emph{main outer loop} and an explicit \emph{inner loop}: at each outer iteration the generalized eigenvalue problem is solved, sensitivity vectors are assembled, and the inner loop then iteratively solves the bound sub-optimization problem~(19) for the \emph{increments} $\Delta\rho_e$ using MMA~\cite{Svanberg1987} or linear programming until those increments converge, after which the design variables are updated as $\rho_e \mathrel{:=} \rho_e + \Delta\rho_e$. No density filter or projection is present in the original formulation.

The implementation used in this study retains the bound variable (scaled as $E_b = \lambda_1 / \lambda_\mathrm{ref}$) and MMA, but departs from the original in three consequential respects. 
First, a Heaviside projection $H(\tilde{x},\beta,\eta)$ with a seven-level continuation schedule $\beta \in \{1,2,4,8,16,32,64\}$ advancing every 40 outer iterations is superimposed on the density field, and a grayness penalty $\gamma(\beta/\beta_{\max}) \cdot N_\mathrm{el}^{-1} \sum_e 4\tilde{\rho}_e(1 - \tilde{\rho}_e)$ is added to the MMA objective. This introduces one forward and one adjoint convolution pass per iteration for the density filter, together with one additional adjoint pass through the Heaviside Jacobian for the penalty gradient.
Second, the inner convergence loop for increments is collapsed to a single MMA call per outer iteration on the absolute design variables; the dual Newton solver solves the separable convex approximation to optimality in one shot but without the explicit increment-convergence check of the original inner loop. 
Third---and most consequentially for runtime---a \emph{trial eigensolve} is performed immediately after every MMA update: the trial physical field is assembled into new $\mathbf{K}$ and $\mathbf{M}$ matrices, and $\min(J,3)$ modes are recomputed to reject MMA steps that decrease $\omega_1$ by more than $1\,\%$. This effectively doubles the number of eigensolves per outer iteration relative to the original scheme.

Both the original and the present implementation employ the implicitly restarted Arnoldi method (\textsc{arpack} via \texttt{eigs(\ldots,\,\textquotesingle SM\textquotesingle)}) with shift-and-invert to extract the $J = 3$ lowest eigenpairs. Shift-and-invert requires a sparse Cholesky factorization of $\mathbf{K}$ at each \texttt{eigs} invocation; because $\mathbf{K}$ changes at every iteration, this factorization is
performed \emph{twice} per outer iteration in the present code. For the quasi-banded beam-like meshes considered here, factorization cost scales as $O(N_\mathrm{dof}^{\,1.5})$ and dominates the per-iteration budget.

Regarding sensitivity analysis, the original formulation handles repeated eigenvalues through an $N\times N$ sub-eigenvalue problem~(Eq.~12 of the reference), whose solution yields the directional derivatives of the degenerate cluster via generalized gradient vectors $\mathbf{f}_{sk}$.  The present implementation instead applies the standard unimodal formula $(\lambda_j)'_{\rho_e} = \boldsymbol{\varphi}_j^{\mathrm{T}}
 (\mathbf{K}'_{\rho_e} - \lambda_j \mathbf{M}'_{\rho_e})\boldsymbol{\varphi}_j$ independently to each of the $J$ modes, omitting off-diagonal terms; each elemental sensitivity is then propagated through the Heaviside Jacobian and the adjoint filter, requiring $J$ further convolution passes.

These deviations collectively determine the observed performance ratios. The Yuksel dynamic code~\cite{Yuksel2025} employs an OC bisection update ($O(N_\mathrm{el})$ per bisection step, $\approx\!15$ steps per iteration) and performs a single eigensolve per outer iteration, with no projection chain and no penalty-gradient filter pass.  The MMA dual Newton solver in the Olhoff implementation costs $O(m \cdot N_\mathrm{el})$ with $m = J + 2 = 5$ constraints, and the trial eigensolve doubles the sparse-factorization overhead. Taken together, these contributions yield a per-iteration cost approximately $7$--$8\times$ higher for the Olhoff approach, as observed in Table~1 (e.g.\ $0.38$ vs.\ $0.05$~s/iter at
$320{\times}40$).  Because the Olhoff implementation terminates in $\approx\!300$ outer iterations against $\approx\!700$ for the Yuksel dynamic code, the net total runtime ratio is $300/700 \times 7.5 \approx 3.2$--$3.5$, consistent with the measured factor of $3.5$ across all four mesh sizes.  Both methods exhibit an empirical per-iteration scaling of roughly $O(N_\mathrm{el}^{\,1.3})$ ($0.08 \to 0.58$~s/iter and $0.01 \to 0.08$~s/iter across the mesh range), confirming that sparse Cholesky factorization dominates over the $O(N_\mathrm{el})$ filter and sensitivity-assembly costs.

Memory consumption follows directly from the MMA storage requirements: the implementation retains the two MMA history iterates $\mathbf{x}^{(k-1)}$,$\mathbf{x}^{(k-2)}$, the asymptote vectors $\ell_i$, $u_i$, and the gradient matrix $\mathbf{dfdx}$ of size $m \times (N_\mathrm{el}{+}1)$; moreover, the trial eigensolve requires that $\mathbf{K}_f$ and $\mathbf{M}_f$ remain allocated simultaneously for both the current and trial fields.  At $400{\times}50$ ($N_\mathrm{el} = 20{,}000$) this explains the observed $384$~MB peak against $123$~MB for the Yuksel approach.

The factor of $80$--$825\times$ reported in Table~1 of~\cite{Yuksel2025} between their proposed static method and their standard dynamic code is not directly comparable to the $3.5\times$ ratio observed here, for two reasons. Algorithmically, their proposed static approach avoids eigenvalue computation \emph{entirely} during the iteration by substituting compliance minimization under a design-dependent inertial load for the eigensolve, whereas both approaches compared in our study retain a full eigensolve (Olhoff: twice, Yuksel dynamic: once) per iteration.
Practically, their reference machine operates at $2.29$~GHz with a 64-core Xeon configuration, while the present results were obtained on an ARM64 platform with Apple Accelerate sparse solvers, leading to substantially different absolute factorization throughputs that preclude direct comparison of absolute timings.

%\subsection{Computational implementation of the proposed methods}
\subsection{The proposed approach: algorithm, efficiency, and practical merits}

The proposed method replaces the dynamic eigenvalue optimization loop with a single initial eigensolve followed by an entirely static iteration.  At initialization, the generalized eigenvalue problem $\mathbf{K}_0\boldsymbol{\varphi}_1 =
\omega_1^2\mathbf{M}_0\boldsymbol{\varphi}_1$ is solved once on the uniform density field $\rho_e^{(0)} = \alpha$ to obtain the fundamental eigenpair $(\omega_1,\boldsymbol{\varphi}_1)$.  The mode shape and eigenvalue are then held \emph{fixed} for the duration of the optimization.  At each subsequent iteration the density-dependent mass matrix $\mathbf{M}(\mathbf{x})$ is assembled under SIMP interpolation, and the equivalent static inertial load $\mathbf{f} = \omega_1^2\,\mathbf{M}(\mathbf{x})\boldsymbol{\varphi}_1$ is evaluated.  The optimization sub-problem reduces to standard compliance minimization,
$\min\,\mathbf{u}^\mathrm{T}\mathbf{K}\mathbf{u}$ subject to $\mathbf{K}\mathbf{u}=\mathbf{f}$ and the volume constraint, updated by the OC bisection rule.  The load sensitivity term $\partial\mathbf{f}/\partial x_e$ is omitted, leaving a purely stiffness-based sensitivity $\partial c/\partial x_e = -p\,x_e^{p-1}(E_0 - E_\mathrm{min})\,\mathbf{u}_e^\mathrm{T}\mathbf{K}_0\mathbf{u}_e$ that is inexpensive to compute and straightforward to filter.  A single eigensolve is performed at termination to verify the achieved fundamental frequency.

From a computational standpoint, each iteration requires one sparse  factorization of $\mathbf{K}_f$ (to solve the linear system
$\mathbf{K}_f\mathbf{u}_f = \mathbf{f}_f$), one mass-matrix assembly, one sensitivity evaluation, and one OC bisection sweep — identical in structure and cost to a standard compliance minimization step. Because the Krylov iterations of \textsc{arpack} are absent, the per-iteration wall time matches that of the Yuksel code (e.g.\ $0.05$~s/iter at $320{\times}40$) while the number of outer iterations is markedly lower: the fixed-mode gradient provides a stable and consistent descent direction from the first iteration, allowing convergence in $259$--$500$ iterations against $622$--$700$ for the Yuksel  code across the four meshes.  The net effect is a total runtime that is consistently the lowest of the three methods: at $320{\times}40$ the proposed approach completes in $13.2$~s against $33.1$~s for the Yuksel  code and $114.4$~s for the Olhoff implementation, representing speedups of $2.5\times$ and $8.7\times$ respectively.  At $400{\times}50$ the advantage over the Yuksel code narrows to $1.2\times$ because the Yuksel iteration count is capped and the per-iteration costs converge, yet the proposed method remains the
fastest throughout.

Memory consumption of the proposed approach is minimal and practically mesh-independent at small scales ($\approx\!4$~MB at $160{\times}20$ and $240{\times}30$), growing to $21$~MB and $112$~MB at the two largest meshes — comparable to the Yuksel  code and three to four times lower than the Olhoff implementation.  The absence of MMA history arrays ($\mathbf{x}^{(k-1)}$, $\mathbf{x}^{(k-2)}$, asymptote vectors, and the $(J{+}2)\times(N_\mathrm{el}{+}1)$ gradient matrix) accounts for most of
the difference relative to the Olhoff approach.

Despite the algorithmic simplification, the fundamental frequencies achieved by the proposed method are in close agreement with those produced by the Yuksel code, as confirmed by the final eigensolve performed after convergence.  This is consistent with Rayleigh's principle: because the initial uniform design is a reasonable starting estimate of the optimal mode shape (both are unimodal for the beam problems considered), fixing $\boldsymbol{\varphi}_1$ empirically observed to produce only minor deviations in the eigenvalue approximation throughout the iteration.  In engineering practice this trade-off — a marginal reduction in the attained frequency in exchange for a substantially shorter computation — is frequently acceptable,
particularly in preliminary design stages or when rapid exploration of multiple boundary-condition configurations is required.

Beyond numerical performance, the simplicity of the proposed method constitutes a practical advantage in its own right.  The algorithm requires no continuation schedule, no bound variable, no multiple-eigenvalue sub-problem, and no MMA solver; it reduces to a verbatim extension of a standard compliance minimization code. Implementation effort is minimal, the
parameter space is small (penalization exponent, filter radius, move limit, and convergence tolerance), and the code path is deterministic and easily debuggable.  These properties lower the barrier to adoption in custom in-house finite-element environments where integrating a full MMA solver or managing Heaviside continuation would represent a significant development
overhead.

\section{Conclusions}

This paper has presented an efficient method for the topology optimization of continuum structures aimed at maximizing target natural frequencies. The proposed approach addresses the primary computational bottleneck in dynamic optimization—the repeated solution of the generalized eigenvalue problem—by replacing it with the solution of a single static design-dependent optimization. This strategy significantly reduces the computational cost compared to established nested iterative procedures, such as those proposed by Du and Olhoff~(2007)~\cite{Du2007}. Furthermore, while our approach shares a similar underlying principle with the recent work of Yuksel and Yilmaz~(2025)~\cite{Yuksel2025}, it distinguishes itself by identifying approximate optimal topologies through a single static problem rather than a two-step sequential procedure, thereby further enhancing algorithmic efficiency.

The effectiveness and efficiency of the proposed method were validated through a performance comparison across various mesh sizes, as summarized in Table 1. The numerical results demonstrate that the proposed method consistently outperforms the standard eigenvalue-based approach and is highly competitive with other fast surrogate methods. For instance, at a mesh size of 320 × 40, the proposed method achieved convergence in only 13.2 seconds, making it approximately 8.6 times faster than the Olhoff formulation (114.4 s) and 2.5 times faster than the Yuksel approach (33.1 s). Additionally, the proposed method exhibits superior memory management; for the largest mesh considered (400 × 50), it required only 112 MB of RAM, a significant reduction compared to the 384 MB required by the Olhoff implementation.

The results indicate that the run time per iteration for the proposed method remains remarkably low and scales efficiently as the mesh resolution increases. This aligns with the broader goals of achieving multigrid-like efficiency in large-scale structural optimization, as discussed in recent literature~\cite{Ferrari2018}. By avoiding the high cost of modal analysis at every step, the current method provides a robust and practical tool for the rapid design of vibration-resistant structures. Future work will focus on extending this single-step static surrogate approach to handle multiple and repeated eigenfrequencies, as well as its application to large-scale three-dimensional optimization problems~\cite{Huang2025, Deng2024}.

\bibliographystyle{unsrtnat}
\bibliography{literature}


\end{document}
